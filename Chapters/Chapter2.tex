\chapter{Background in Fault Detection, Isolation and Recovery}

In this chapter, we will discuss the theoretic fundamentals of Fault Detection, Isolation and Recovery (hereafter, ``FDIR"). We will look at how behavior of complex systems can be modeled in such a way that undesirable states can be clearly defined and detected. We will look at more modern, advanced techniques for this analysis. We'll also spend some time talking about analytical as well as human-centered issues with current techniques, in order to motivate the subsequent work within this paper.

\section{Goals and Definitions}

% Terms to define: fault, detect, isolate, recover, residual, operational modes, fault levels

\cite{schwabacher2008pre}
\cite{dearden2004real}

fault levels
\cite{tipaldi2014spacecraft}


\section{Model-Based FDIR}

\cite{willsky1976survey}


\subsection{System Model-Based FDIR}

\cite{hwang2010survey}


\subsection{Rule-Based FDIR}


\cite{schwabacher2008pre}


\subsection{Residual Generation}


mention Thresholds

\section{Fault Filtering and State Observation}


\subsection{Full-State Observer Fault Filters}


\subsection{Parity-Space Filtering}


\subsection{Kalman Filtering for FDIR}

\cite{larson2002model}
\cite{washington2000board}

Also, particle filters:

\cite{dearden2004real}

\section{Redundancy and Voting Systems}


\section{Advanced Techniques for Fault Detection}

\subsection{Fault Syndromes}

\cite{hammett1991application}


\subsection{Machine Learning and Classification}

SVMs: \cite{lin2006fault}
HMMs for state learning/classification: \cite{aycard2000state}
Bayes: \cite{holsti2001towards} \cite{paakko2001bayesian}

-talk about their downsides, like needing to have a lot of training data, and hardness to debug

\subsection{Distributed Fault Control}

\cite{castel2006fdir}


\subsection{Common FDIR Issues}

\cite{kurien2010intrinsic}

-write about problems with root cause analysis

-lead into the higher-level analysis of correlation between faults, and my theories about how this can be effective
-Human-Centered Considerations for FDIR