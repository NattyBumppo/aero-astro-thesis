\chapter{Background in Fault Detection, Isolation and Recovery}

In this chapter, we will discuss the theoretic fundamentals of Fault Detection, Isolation and Recovery (hereafter, ``FDIR"). We will look at how behavior of complex systems can be modeled in such a way that undesirable states can be clearly defined and detected. We will look at more modern, advanced techniques for this analysis. We'll also spend some time talking about analytical as well as human-centered issues with current techniques, in order to motivate the subsequent work within this paper.

\section{Goals and Definitions}

First, we will define a handful terms in order to precisely discuss FDIR problems.

A \textbf{fault} 

A fault is \textbf{detected}

The process of fault \textbf{isolation}


After a fault is isolated, the final step for an autonomous system is to \textbf{recover}

System behavior may be modeled as a finite state machine of \textbf{operational modes},

A system state or subset of values can be said to be \textbf{nominal} when

it can be described as \textbf{anomalous} when 

also described as \textbf{out of family}

System models may include ideal or expected behavior for a given operation mode, and a method for comparing the current state to the expected mode can be expressed as a \textbf{residual}

Furthermore, faults can be divided into \textbf{fault levels} \cite{tipaldi2014spacecraft}.


\cite{schwabacher2008pre}
\cite{dearden2004real}





\section{Model-Based FDIR}

\cite{willsky1976survey}


\subsection{System Model-Based FDIR}

\cite{hwang2010survey}


\subsection{Rule-Based FDIR}


\cite{schwabacher2008pre}


\subsection{Residual Generation}


A simpler version of fault residual comparison can be found in the setting of upper and lower bounds to describe the range of nominal values for a state. This ``threshold-based" system is the easiest variant of fault detection to implement, and is an important part of many modern FDIR systems \cite{walker1979fault}.

\section{Fault Filtering and State Observation}


\subsection{Full-State Observer Fault Filters}


\subsection{Parity-Space Filtering}


\subsection{Kalman Filtering for FDIR}

\cite{larson2002model}
\cite{washington2000board}

Also, particle filters:

\cite{dearden2004real}

\section{Redundancy and Voting Systems}


\section{Advanced Techniques for Fault Detection}

\subsection{Fault Syndromes}

\cite{hammett1991application}


\subsection{Machine Learning and Classification}

SVMs: \cite{lin2006fault}
HMMs for state learning/classification: \cite{aycard2000state}
Bayes: \cite{holsti2001towards} \cite{paakko2001bayesian}

-talk about their downsides, like needing to have a lot of training data, and hardness to debug

\subsection{Distributed Fault Control}

\cite{castel2006fdir}


\subsection{Common FDIR Issues}

\cite{kurien2010intrinsic}

-write about problems with root cause analysis

-lead into the higher-level analysis of correlation between faults, and my theories about how this can be effective
-Human-Centered Considerations for FDIR