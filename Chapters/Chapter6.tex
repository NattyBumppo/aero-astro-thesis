\chapter{Intermediate Results and Reassessment}

This chapter discusses some of the learnings from our intermediate testing, and how we used them to shape our next steps for correlative assessment techniques.

\section{Field Test Results}

Our field tests went swimmingly, and we returned to Japan with excellent results and ideas for improvement. Though our tests went much better than we hoped, they were not without a small number of issues. We uncovered several minor bugs within the ground station, which were quickly fixed on-the-spot, thanks to the rapid iterative capabilities afforded by the Unity Game Engine. The tests---especially the act of watching others operate the ground station---yielded useful data about which features were used the most, and which were all but ignored entirely.

We found the ``immersive viewing" feature to be immensely helpful for gaining situational awareness, and proved its efficacy in testing where engineers placed the rover in a precarious situation, then asked us over radio to use the telemetry and visual data to give them a detailed description of the current situation. The ``connectivity map" also proved extremely useful in quickly informing us of issues with the rover, as did the overall fault detection system.

A previous feature for displaying detailed data about downloaded files turned out to be mostly unnecessary and unused, despite the large amount of development time that had gone into its creation. At the same time, fault information was shown to be lacking, and there were several instances of users accidentally ignoring unsafe rover conditions. In response to these observations, the downloaded file data display was removed, and the space that it made available was used for providing more detailed fault information. Also, fault detection and display rules were rethought and improved, to enhance visibility of anomalous states. This is only a small sampling of the numerous improvements that were made in response to field test results.

\section{Usability Test Results}

Usability testing yielded many results that were useful for setting subsequent development priorities. Overall, all participants were able to deduce the general course of events undergone by the rover, and at the end were able to narrate a story which resembled the intended one outlined above.

Users remarked that they found the Copilot Screen's telemetry display the most useful, and they spent the majority of their time monitoring telemetry data on this screen. Most participants used the data review feature heavily while on this screen, rewinding and fast-forwarding through time and watching displayed telemetry values change as they did so. Users remarked that this feature was easy to use and excellent for reviewing the flow of telemetry. However, multiple users expressed a desire for time series plots of data channels, to better see the history of data at a glance.

Faults were generally very visible, and users commented that they found the additional fault information very helpful when trying to understand the meanings of various fault states. However, I did observe that tunnel vision was occasionally a problem for users; too much attention on one specific UI component, such as the correlation map or displayed telemetry on the Copilot Screen, seemed to be responsible for delays of up to 30 seconds in reacting to critical fault events. It seems that a better method is needed to redirect the user's attention in certain cases; I am currently considering blinking panels and sound cues, among other possibilities.

Multiple users commented on the difficulty of understanding telemetry for channels whose meaning they did not understand. (Many of the subsystems have very specific data channels whose meaning is not well understood to anyone except the designers of those systems.) Results indicate that the detailed fault information mentioned above were effective in eliminating much of the confusion about specific faults; however, I believe that adding information that leads to a better understanding of individual telemetry channels would result in less user confusion and could improve the effectiveness of human telemetry monitoring. I am currently collecting detailed information about data channels from subsystem engineers to incorporate this data into the interface.

The usability test uncovered many issues with the correlation map in its current form. Many of the users commented that they did not understand the proper way to use it to analyze data (although they understood the basic idea of the visualization). Users requested better, more intuitive spatial organization of data channel pairs, and the ability to more easily filter channels of interest and ones pertaining to faults. I received comments that additional training sessions might be beneficial. Nearly everyone expressed an interest in using it, but the performance of those who endeavored to analyze faults with this tool showed evidence of a need for automated simplification to reduce stimuli, and to come up with better ways to train users and to lead them to useful conclusions.

A few other minor issues came up, which should be easier to address but which I had not foreseen until performing the testing. Some users had difficulties with transition lag between screens (there is a lag of approximately a second between screen transitions due to a need to load in resources). I am looking into performance enhancements and/or loading screens to fix this issue. I also received the feedback that a more visible speed/RPM indicator for the rover would be helpful, as speed is one of the most important aspects of the rover as it operates, and this data is easily overlooked in the midst of other types of telemetry. I am currently looking into more highly-visible visualizations for RPM, based on car dashboard tachometer and speedometer displays.

\section{Analytical Improvements and Additions}

Phasellus nisi quam, volutpat non ullamcorper eget, congue fringilla leo. Cras et erat et nibh placerat commodo id ornare est. Nulla facilisi. Aenean pulvinar scelerisque eros eget interdum. Nunc pulvinar magna ut felis varius in hendrerit dolor accumsan. Nunc pellentesque magna quis magna bibendum non laoreet erat tincidunt. Nulla facilisi.

Duis eget massa sem, gravida interdum ipsum. Nulla nunc nisl, hendrerit sit amet commodo vel, varius id tellus. Lorem ipsum dolor sit amet, consectetur adipiscing elit. Nunc ac dolor est. Suspendisse ultrices tincidunt metus eget accumsan. Nullam facilisis, justo vitae convallis sollicitudin, eros augue malesuada metus, nec sagittis diam nibh ut sapien. Duis blandit lectus vitae lorem aliquam nec euismod nisi volutpat. Vestibulum ornare dictum tortor, at faucibus justo tempor non. Nulla facilisi. Cras non massa nunc, eget euismod purus. Nunc metus ipsum, euismod a consectetur vel, hendrerit nec nunc.

\section{Visualization Improvements and Additions}

Phasellus nisi quam, volutpat non ullamcorper eget, congue fringilla leo. Cras et erat et nibh placerat commodo id ornare est. Nulla facilisi. Aenean pulvinar scelerisque eros eget interdum. Nunc pulvinar magna ut felis varius in hendrerit dolor accumsan. Nunc pellentesque magna quis magna bibendum non laoreet erat tincidunt. Nulla facilisi.

Duis eget massa sem, gravida interdum ipsum. Nulla nunc nisl, hendrerit sit amet commodo vel, varius id tellus. Lorem ipsum dolor sit amet, consectetur adipiscing elit. Nunc ac dolor est. Suspendisse ultrices tincidunt metus eget accumsan. Nullam facilisis, justo vitae convallis sollicitudin, eros augue malesuada metus, nec sagittis diam nibh ut sapien. Duis blandit lectus vitae lorem aliquam nec euismod nisi volutpat. Vestibulum ornare dictum tortor, at faucibus justo tempor non. Nulla facilisi. Cras non massa nunc, eget euismod purus. Nunc metus ipsum, euismod a consectetur vel, hendrerit nec nunc.