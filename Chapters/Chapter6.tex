\chapter{Intermediate Results and Reassessment}

This chapter discusses some of the learnings from our intermediate testing, and how we used them to shape our next steps for correlative assessment techniques.

\section{Field Test Results}

Our field tests went swimmingly, and we returned to Japan with excellent results and ideas for improvement, but these tests were not without a small number of issues. We uncovered several minor bugs within the ground station, which were quickly fixed on-the-spot. The tests---especially the act of watching others operate the ground station---yielded useful data about which features were used the most, and which were all but ignored entirely.

We found the ``immersive viewing" feature to be immensely helpful for gaining situational awareness, and proved its efficacy in testing where engineers placed the rover in a precarious situation, then asked us over radio to use the telemetry and visual data to give them a detailed description of the current situation. The ``connectivity map" also proved extremely useful in quickly informing us of issues with the rover, as did the overall fault detection system. Some features were determined to be of limited utility and removed.

Fault information was shown to be somewhat sparse for encoding expert understanding of problems, and there were several instances of users accidentally ignoring unsafe rover conditions. In response to these observations, space made available from removing other features was used for providing more detailed fault information (this was an enhancement of the ``Expert fault information" feature discussed in the previous chapter). Also, fault detection and display rules were rethought and tweaked, to enhance visibility of anomalous states.

\section{Usability Test Results}

Usability testing yielded many results that were useful for setting subsequent development priorities. Overall, all participants were able to deduce the general course of events undergone by the rover, and at the end were able to narrate a story which resembled the intended one outlined above.

Users remarked that they found the Copilot Screen's telemetry display the most useful, and they spent the majority of their time monitoring telemetry data on this screen. Most participants used the data review feature heavily while on this screen, rewinding and fast-forwarding through time and watching displayed telemetry values change as they did so. Users remarked that this feature was easy to use and excellent for reviewing the flow of telemetry. However, multiple users expressed a desire for time series plots of data channels, to better see the history of data at a glance.

Faults were generally very visible, and users commented that they found the additional fault information very helpful when trying to understand the meanings of various fault states. However, it was observed that tunnel vision was occasionally a problem for users; too much attention on one specific UI component, such as the correlation map or displayed telemetry on the Copilot Screen, seemed to be responsible for delays of up to 30 seconds in reacting to critical fault events. This observation led to the addition of audio cues to redirect user attention, which has already shown to be effective in informal testing.

Multiple users commented on the difficulty of understanding telemetry for channels whose meaning they did not understand. (Many of the subsystems have very specific data channels whose meaning is not well understood to anyone except the designers of those systems.) Results indicate that the expert fault information feature mentioned above was effective in eliminating much of the confusion about specific faults; however, we believe that adding information that leads to a better understanding of individual telemetry channels would result in less user confusion and could improve the effectiveness of human telemetry monitoring. Knowledge capture efforts are currently underway to collect detailed information about data channels from subsystem engineers to incorporate this data into the interface.

The usability test uncovered many issues with the correlation map in its current form. Many of the users commented that they did not understand the proper way to use it to analyze data (although they understood the basic idea of the visualization). Users requested better, more intuitive spatial organization of data channel pairs, and the ability to more easily filter channels of interest and ones pertaining to faults. We received comments that additional training sessions might be beneficial. Nearly all users expressed an interest in using this feature, but the performance of those who endeavored to analyze faults with this tool showed evidence of a need for automated simplification to reduce stimuli, and to come up with better ways to train users and to lead them to useful conclusions.

A few other minor issues came up which we had not foreseen until performing the testing. Some users had difficulties with transition lag between screens (there is a lag of approximately a second between screen transitions due to a need to load resources). We am looking into performance enhancements and/or loading screens to fix this issue. We also received the feedback that a more visible speed/RPM indicator for the rover would be helpful, as speed is one of the most important aspects of the rover as it operates, and this data is easily overlooked in the midst of other types of telemetry. This led to the development of the visual tachometer RPM visualization discussed in the previous chapter.

\section{Improvements and Additions}

Looking at the rests of this intermediate testing, we were able to identify various targets for further research and improvements in the field of analysis of the fault-related, and correlative data.

Even with the small number of data channels available on the Moonraker rover (~112), the dimensionality was very high for a full corrgram display, and seemed to stretch the visual and attentive capacities of the users who participated in the test. Even with the addition of data filtering features, without focused training of the use of these features, they didn't seem to provide a better experience for users looking for patterns in the correlative data. Though users were able to, from an integrative viewing of telemetry data as shown in the numerical and color-based fault displays, able to come to understand a timeline of the rover events, mental links between associated channels seemed to emerge due to sheer coincidence; remarks from users were along the lines of ``the temperature is increasing... and I see that the voltage is increasing too... maybe they're related." The correlation map, as a way to call out these patterns, did not seem to be as effective as anticipated.

While this data was successfully captured by the analysis, we determined that it was a) buried in the noise of lots of unimportant correlations and b) shown on a larger visual display with too much data to easily visual process in a short amount of time. What's more, c) there was no affordance for users to simultaneously display time across different temporal points.

As such, we determined that it would be of value to iterate on these three points, with the main goals of reducing correlative noise and data dimensionality and of providing a simplified visualization capable of displaying data from multiple time points simultaneously. The next chapter will discuss a few approaches towards this end and their results.