\chapter{Future Work and Conclusion}

In this chapter, we will look at possible future extensions of this work, and conclude by summarizing what we have attempted and accomplished.

\section{Future Work}

We have envisioned, and implemented, several innovations on telemetry visualization techniques within this research. As such, the space of possible applications is very broad, but we will attempt to list a few promising avenues of potential research.

It remains to be explored how the three major telemetry visualization developments within this research can be integrated with each other and with more traditional ground station telemetry displays. Data channels are the consistent presence between our three new types of visualizations and traditional numerical and plotted time series visualizations, so it seems that interactive affordances ought to be designed to allow the user to navigate from one visualization to another using the data channels as portals.

For example, when viewing a time curve correlative visualization, the user might discover an anomalous excursion from a known correlative cluster. (As a further extension, these clusters, or modes, could be defined by experts or learned through training steps.) A UI affordance could provide the ability to find the major correlative components contributing to this deviation from the expected state, producing a list of anomalous correlations. These correlations could be shown on an undirected correlation graph in context with others. An animated corrgram, specifically highlighting the pairwise correlations of interest, could be generated and played back over a time window leading up to the anomalous excursion. Affected channels could also be highlighted on a traditional telemetry display, with appropriate time series visualizations. This is just one foreseeable integrated flow that could lead to more effective root cause analysis; a flow in the opposite logical direction, in which a residual-based fault is detected by the autonomous FDIR system, and applicable correlative data is percolated up to the human operator via our visualizations, is another likely use-case.

Currently, only our animated corrgram is calculated in real-time; there is still a multi-step, manual pipeline for generating undirected correlation graphs and time curve visualizations. These features would need to be integrated into a real-time pipeline to be used within the context of a software ground station, but with modest optimizations, this should be computationally feasible. Real-time generation of these visualizations would allow more than just retrospective data review; real-time data could be analyzed and displayed to human operators in order to identify fault-suggestive states as soon as the relevant telemetry is received.

Another next step for our research, and one which we are already actively pursuing, is the application of our work to a larger number of ongoing problems and datasets. By working with spacecraft teams who are already dealing with high-dimensional telemetry data, we hope to improve the usability and effectiveness of our tools.

Finally, further usability research is needed to assess enhanced performance in understanding system state and fault reasoning for both highly-trained and untrained users using this software. It is highly likely that such research would spark the development of improved visualization tools for human-in-the-loop fault analysis, as the intermediate testing described in Chapter 5 of this thesis already has.

\section{Conclusion}

In this thesis, we studied possible solutions to the difficult problem of root case analysis for complex systems. We began with a review of FDIR systems, and described their shortcomings in order to motivate further work. We looked at correlative measures of complex state data--which can be used to show hard-to-discover links between elements of system state--and showed how this data is traditionally visualized.

We then built an innovative animated corrgram as part of a ground software visualization suite for a real-life space system. We tested this visualization in multiple real-world and simulated scenarios, and showed that it had usability issues with user cognitive load and data discoverability, although the analytical insights that it made possible proved valuable upon inspection.

Iterating further, we described data analysis and visualization techniques inspired by problems which emerged in prior testing. We developed a graph-based visualization showing correlation between different telemetry channels, and showed how this gives a clear understanding to a human operator of the relationships between channels. We also built a visualization of correlative state over time, and showed how system modes and trajectory are clearly visible in this very useful visualization. We discussed how insights gained from these methods of data display help to assist the human-centered side of root cause analysis for system faults. Finally, we discussed exciting potential avenues for future work.

This work has accomplished not only a deep dive into an interesting neighborhood within the city of fault detection, isolation and recovery, but has also succeeded in developing new tools to encourage greater understanding and improved analysis of system behavior. The new tools are innovative and revolutionary, and succeed in giving exciting new insight into system behaviors which were not properly captured by raw state data.

It is the hope of the author that this work finds broad application, and contributes to the state-of-the-art of system FDIR and telemetry analysis within the space industry in the future.