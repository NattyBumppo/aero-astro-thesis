\chapter{Introduction}

Complex, remote-operated systems present many problems to engineers and designers who are conscious of mission safety. Complex systems often have detailed and comprehensive rules for the detection of anomalous conditions, or ``faults," but even the most complex cannot capture the full range of possible anomalies that can occur on a system, especially if false positives are a concern and if the user has not thought of all possible conditions. Visualization of fault conditions can be an even greater problem, owing to issues such as the impracticality of simultaneously displaying data from thousands of telemetry channels, organizing data in a logical and discoverable way, maintaining system reliability in the presence of performance constraints, and leaving screen space for other non-fault-related visualization components and control affordances.

Because of these difficulties, root cause analysis can be a very long and difficult task. There are several historical cases of major system anomalies that have required very long periods of concentrated, manual telemetry data analysis (and, in some of the most catastrophic cases, post-disassembly hardware analysis) in order to piece together the root cause for anomalies. Some examples include:

\begin{itemize}

\item On October 28th, 2014, the Orbital Sciences ``Antares" rocket suffered a catastrophic failure 6 seconds after launch, experiencing a large explosion which destroyed its cargo, which had been bound for the International Space Station. Orbital Sciences immediately launched an investigation to determine the cause, but preliminary root cause data loosely linking the mishap to a failure of one of the AJ26 engines was not publicly indicated until November 5th, and a final root cause assessment has still not, to this date, been released \cite{spaceflight_antares}. An independent review team within NASA evaluated telemetry data, historical data and hardware samples, beginning in November 2014, and roughly a year later, issued a report that was still unable to provide a clear root cause for the mishap, instead linking it to three likely causes, all involving the AJ26 engine which initially exploded \cite{nasa_orb3}.

\item On July 28th, 2015, SpaceX's ``Falcon 9" rocket, carrying the ``CRS-7" payload delivery up to the International Space Station, experienced an overpressure event in the second stage liquid oxygen tank, causing rapid unscheduled disassembly and failure of the mission. SpaceX engineers, working with NASA and the Air Force, began intensively examining system telemetry from the event. Despite SpaceX's well-known history of transparency about anomalous events and engineering challenges, the root cause of flawed second-stage helium system strut was not publicly identified until nearly a month later, on July 20th \cite{spacex_crs7}.

\item On December 4rd, 2015, the PROCYON interplanetary cubesat, developed by the University of Tokyo and JAXA, went completely ``dark," ceasing to provide any telemetry data at all. As of February of 2016, attempts to analyze previously received telemetry data in order to gain insight into the cause of the anomaly, and possible ideas for recovery, continue, but no root cause has been able to be determined \cite{procyon_status}.

\end{itemize}

As is shown by the cases above, the root cause diagnosis process is very difficult, and can take weeks to months to complete. It involves intense scrutiny of potentially thousands of data channels, and often the only comprehensive understanding of how these data channels relate to each other is encoded in human ``tribal knowledge." Although the examples above are extreme ones, root cause diagnosis can be extremely valuable even with trivial anomalies for gaining a better understanding of system properties and subsystem connections, and the tools to do so that are currently in use are inadequate for the task. Having seen this problem first-hand in the space industry, we set out to examine the space of possible tools that could begin to tackle this problem.

In this paper, we will examine some possible solutions to the long-standing problem of root cause analysis for complex, remotely monitored systems. The paper starts with a review of schemes for identification of irregular telemetry via a typical fault detection models, and describes the necessity to characterize anomalous conditions in a more in-depth way than traditional fault detection algorithms allow, in order to identify root causes for anomalies, or to predict the occurrence of anomalous conditions ahead of time. We will point out some of the issues that commonly occur with time series data on multiple channels which can make it difficult to both analyze and visualize.

Next, we will examine traditional methods of analyzing connections between sets of time series data. We will see how solutions to some of the aforementioned issues are provided by these analytical techniques. We will also examine a number of visualization techniques that have been used in the past to show connections within correlation data.

We will propose a data visualization technique, based on an animated adaptation of statistical correlation matrices. To assess the efficacy of this visualization, we will apply it to simulated data from a sample system, and will show test results when users are given an implementation of this visualization and asked to use it to gain insight into events during a simulated scenario. We will discuss some of the downsides of our techniques that were uncovered by testing.

Next, we will iterate and propose adaptations to address some of the issues in the first round of testing. We will look at analytical improvements as well as new visualizations, borrowing from recent research in the field of temporal data visualization, and will evaluate them for their efficacy.

Finally, we will present our conclusions about the employed analysis and visualization techniques, and will propose avenues for further research.